% Some commands used in this file
\newcommand{\package}{\emph}

\chapter{Introduction}\label{chapter:intro}
For hundreds of years, people have been using signatures, typically to bind a person to a contract. Since the digital age, digital signatures have appeared which basically have the same ulterior motivates, i.e., binding two objects to each other. Typical examples are binding a name to a person or a name to an IP-address like it is done in the \emph{Domain Name Security Extensions} (DNSSEC). Such signing mechanisms most likely use a \emph{public key infrastructure} (PKI). The very crucial part in every system that uses PKI is to keep the private/signing key secure. This not always an easy task to solve, as a system where the signing key is stored might be compromised or even stolen physically. One goal of this thesis is to figure out if it is possible to build a rather simple signing system that is located in a totally isolated and safe environment such that it can be used to sign assertions (e.g., the mapping of a name to an address) over an air-gapped channel. Based on the idea of \citet{castle}, the air-gapped channel will consist of two monitors standing in front of each other and two cameras pointing to the opposite monitor. The reason why we want to use such a visual air-gap is that one can easily monitor and audit what happens on the screens, i.e., on the channel. Furthermore, the system should be of relatively low cost and yet effective, since this system will mainly be used to sign assertions from the RAINS protocol used in the new Internet architecture SCION \cite{scion_book}, where a lot of requests have to be handled.

Sometimes, managing secret keys like the signing key needs involvement of a human being. While a protocol implemented in a program simply follows the steps that it needs to do, a human can make mistakes, e.g., leave some step out or lose some key. The \emph{control-plane PKI} of SCION \cite{scion_book} uses an \emph{online} and \emph{offline} asymmetric key pair. While the online key is used for frequent mechanisms, the offline keys are used for infrequent and safety critical operations like adding or removing a core AS to the \emph{Trust Root Configuration} (TRC). Unlike the automated use of online keys, the use of offline keys needs interaction with an administrator. Since no human is perfect, even a trained administrator can make mistakes. For this reason, the second goal of this thesis is to formally verify key ceremonies based on the work of \citet{basin2016modeling}. By modeling the ceremonies including human interaction, we want to verify a safe and secure execution of the protocol despite the existence of fallible humans.

In the first part of this thesis we want to describe the signing system, how it is composed, how it works and what is needed to operate it. This is followed by discussing the key management of the signing system in \autoref{chapter:key_management}. In \autoref{chapter:formal_verification} we start describing the protocol of a ceremony, followed by modeling and verifying it.